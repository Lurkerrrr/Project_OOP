\chapter{Podsumowanie}

W ramach projektu zrealizowano wszystkie zaplanowane etapy, obejmujące analizę, projektowanie oraz implementację aplikacji. Ostatecznym rezultatem jest funkcjonalna aplikacja desktopowa do zarządzania sklepem z częściami samochodowymi, wykorzystująca technologię Windows Forms i backend oparty na REST API. Aplikacja została zintegrowana z systemem kontroli wersji Git, a jej kod źródłowy umieszczono w publicznym repozytorium.

\section{Zrealizowane prace}
\begin{enumerate}
    \item \textbf{Analiza wymagań}: Zidentyfikowano kluczowe funkcjonalności aplikacji, takie jak zarządzanie dostawcami, produktami i kategoriami.
    \item \textbf{Projektowanie aplikacji}: Przygotowano strukturę aplikacji, diagramy klas i przepływu danych.
    \item \textbf{Implementacja serwisów i kontrolerów}: Zrealizowano backend z wykorzystaniem .NET.
    \item \textbf{Interfejs użytkownika}: Stworzono GUI w technologii Windows Forms, zapewniające intuicyjność i wygodę użytkowania.
    \item \textbf{Testowanie i optymalizacja}: Przeprowadzono testy jednostkowe i integracyjne oraz optymalizację wydajności.
    \item \textbf{Dokumentacja}: Przygotowano dokumentację projektu, w tym diagramy Gantta i szczegółowy opis funkcjonalności.
\end{enumerate}

\section{Możliwości dalszego rozwoju}
Aplikacja jest modułowa i pozwala na łatwy rozwój. Możliwe kierunki rozwoju to:
\begin{itemize}
     \item Rozbudowa bazy danych o nowe modele i specyficzne funkcje w endpointach API.
    \item Migracja na aplikację webową z użyciem nowoczesnych frameworków.
    \item Rozszerzenie funkcjonalności o generowanie raportów, analizy danych czy powiadomienia email.
    \item Stworzenie wersji mobilnej z wykorzystaniem Xamarin lub Flutter.
    \item Ulepszenie UX/UI z wykorzystaniem nowoczesnych bibliotek graficznych.
\end{itemize}

\section{Podsumowanie końcowe}
Projekt został zrealizowany zgodnie z założeniami i harmonogramem. Aplikacja spełnia wymagania i jest gotowa do dalszego rozwoju. Modularność i użyte technologie umożliwiają łatwą adaptację do zmieniających się potrzeb i rozbudowy funkcjonalności, co stanowi solidną podstawę do wdrożenia w środowisku produkcyjnym.
