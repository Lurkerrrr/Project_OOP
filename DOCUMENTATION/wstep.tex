\chapter*{Wstęp}

Projekt „CarParts” ma na celu stworzenie aplikacji służącej do zarządzania częściami samochodowymi, ich kategoriami oraz dostawcami. Głównym celem jest zapewnienie użytkownikom intuicyjnego i prostego w obsłudze interfejsu umożliwiającego zarządzanie informacjami o częściach samochodowych w sposób efektywny i uporządkowany.

Aplikacja wykorzystuje technologie takie jak C\# i ASP.NET Core, co zapewnia stabilność oraz możliwość łatwego skalowania projektu. System pozwala na wykonywanie operacji CRUD na głównych encjach, takich jak Kategorie, Produkty i Dostawcy, a także zapewnia przejrzysty interfejs użytkownika oparty na formularzach graficznych.

 \section*{Założenia projektu}
\begin{enumerate}
    \item Projekt o nazwie \textbf{CarParts} to aplikacja do zarządzania częściami samochodowymi, kategoriami oraz dostawcami, zbudowana w technologii \textbf{C\#} i \textbf{ASP.NET Core}.
    \item Aplikacja umożliwia realizację operacji CRUD (Create, Read, Update, Delete) na encjach: \textbf{Category}, \textbf{Product}, i \textbf{Supplier}.
    \item Interfejs użytkownika jest zbudowany z wykorzystaniem formularzy graficznych \textbf{GUI Forms}, co zapewnia intuicyjne zarządzanie danymi przez użytkownika.
    \item Struktura projektu jest warstwowa, co obejmuje warstwy: \textbf{Controllers}, \textbf{Services}, \textbf{Models}, \textbf{Helpers}, i \textbf{Middleware}.
    \item Dane aplikacji są przechowywane w plikach JSON, zarządzanych przy użyciu klasy \textbf{FileStorageHelper}.
    \item Globalne przetwarzanie wyjątków zostało zaimplementowane w klasie \textbf{ExceptionHandlingMiddleware} dla stabilności systemu.
\end{enumerate}

\section*{Cele projektu}
\begin{enumerate}
    \item Stworzenie skalowalnej i łatwej w utrzymaniu aplikacji do zarządzania częściami samochodowymi.
    \item Zapewnienie użytkownikom możliwości zarządzania kategoriami, produktami oraz dostawcami w sposób intuicyjny i wydajny.
    \item Ułatwienie zarządzania danymi za pomocą graficznego interfejsu użytkownika z obsługą operacji CRUD.
    \item Wdrożenie dobrej praktyki projektowania, opartej na podejściu warstwowym, aby oddzielić logikę biznesową od prezentacji i danych.
    \item Zastosowanie narzędzi takich jak \textbf{Swagger} dla automatycznej dokumentacji API.
\end{enumerate}
\newpage

