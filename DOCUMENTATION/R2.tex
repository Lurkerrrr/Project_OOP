% ********** Rozdział 2 **********
\chapter{Opis struktury projektu}

\section{Struktura i Opis Techniczny}
System został zaprojektowany w oparciu o architekturę warstwową, składającą się z następujących komponentów:
\begin{itemize}
    \item Warstwa Kontrolerów: Obsługuje zapytania HTTP i deleguje logikę do odpowiednich usług.
    \item Warstwa Usług: Implementuje logikę biznesową, operując na danych i koordynując interakcje z bazą danych.
    \item Warstwa Modeli: Definiuje struktury danych używane w systemie.
    \item Baza Danych: Przechowuje dane systemowe w formacie JSON, korzystając z plików jako persystencji.
\end{itemize}

\section{Język, Narzędzia i Wymagania Sprzętowe}
\subsection{Język Programowania}
System został zaimplementowany w języku \textbf{C\#} z wykorzystaniem frameworka \textbf{ASP.NET Core}.

\subsection{Narzędzia}
\begin{itemize}
    \item IDE: Microsoft Visual Studio
    \item Biblioteki: Newtonsoft.Json (do operacji na danych JSON)
    \item Serwer: Kestrel wbudowany w ASP.NET Core
    \item Testowanie: Swagger UI dla testowania API
\end{itemize}

\subsection{Minimalne Wymagania Sprzętowe}
\begin{itemize}
    \item Procesor: Dual-core 2 GHz lub lepszy
    \item RAM: 4 GB
    \item Dysk: 100 MB wolnego miejsca
    \item System Operacyjny: Windows 10 lub nowszy
\end{itemize}

\section{Zarządzanie Danymi i Baza Danych}
Dane w systemie przechowywane są w formacie JSON w plikach:
\begin{itemize}
    \item \texttt{categories.txt} - kategorie produktów
    \item \texttt{products.txt} - produkty
    \item \texttt{suppliers.txt} - dostawcy
\end{itemize}

Operacje na danych realizowane są za pomocą klasy \texttt{FileStorageHelper}, która umożliwia odczyt i zapis danych JSON.

\section{Hierarchia Klas i Kluczowe Metody}
Hierarchia klas została zaprojektowana w sposób modularny i rozszerzalny. Kluczowe klasy to:

\subsection{Klasa \texttt{Category}}
Definicja kategorii produktów.
\begin{itemize}
    \item \textbf{Właściwości:}
    \begin{itemize}
        \item \texttt{Id} - identyfikator kategorii
        \item \texttt{Name} - nazwa kategorii
        \item \texttt{Description} - opis kategorii
    \end{itemize}
\end{itemize}

\subsection{Klasa \texttt{CategoryService}}
Implementacja logiki biznesowej dla kategorii.
\begin{itemize}
    \item \texttt{GetAllCategories()} - zwraca wszystkie kategorie.
    \item \texttt{GetCategoryById(int id)} - zwraca kategorię na podstawie \texttt{Id}.
    \item \texttt{AddCategory(Category category)} - dodaje nową kategorię.
    \item \texttt{UpdateCategory(int id, Category category)} - aktualizuje istniejącą kategorię.
    \item \texttt{DeleteCategory(int id)} - usuwa kategorię.
\end{itemize}

\subsection{Klasa \texttt{Product}}
Definicja produktu.
\begin{itemize}
    \item \textbf{Właściwości:}
    \begin{itemize}
        \item \texttt{Id} - identyfikator produktu
        \item \texttt{Name} - nazwa produktu
        \item \texttt{Price} - cena produktu
        \item \texttt{Code} - kod produktu
    \end{itemize}
\end{itemize}

\subsection{Klasa \texttt{ProductService}}
Logika biznesowa dla produktów.
\begin{itemize}
    \item \texttt{GetAllProducts()} - zwraca wszystkie produkty.
    \item \texttt{AddProduct(Product product)} - dodaje nowy produkt.
    \item \texttt{UpdateProduct(int id, Product product)} - aktualizuje istniejący produkt.
\end{itemize}

\subsection{Klasa \texttt{Supplier}}
Definicja dostawcy.
\begin{itemize}
    \item \textbf{Właściwości:}
    \begin{itemize}
        \item \texttt{Id} - identyfikator dostawcy
        \item \texttt{Name} - nazwa dostawcy
        \item \texttt{ContactInfo} - informacje kontaktowe
        \item \texttt{Address} - adres
    \end{itemize}
\end{itemize}

\subsection{Klasa \texttt{SupplierService}}
Logika biznesowa dla dostawców.
\begin{itemize}
    \item \texttt{GetAllSuppliers()} - zwraca wszystkich dostawców.
    \item \texttt{AddSupplier(Supplier supplier)} - dodaje nowego dostawcę.
    \item \texttt{UpdateSupplier(int id, Supplier supplier)} - aktualizuje dane dostawcy.
\end{itemize}

% ********** Koniec rozdziału **********
